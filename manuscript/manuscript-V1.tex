% Options for packages loaded elsewhere
\PassOptionsToPackage{unicode}{hyperref}
\PassOptionsToPackage{hyphens}{url}
\PassOptionsToPackage{dvipsnames,svgnames,x11names}{xcolor}
%
\documentclass[
]{article}

\usepackage{amsmath,amssymb}
\usepackage{setspace}
\usepackage{iftex}
\ifPDFTeX
  \usepackage[T1]{fontenc}
  \usepackage[utf8]{inputenc}
  \usepackage{textcomp} % provide euro and other symbols
\else % if luatex or xetex
  \usepackage{unicode-math}
  \defaultfontfeatures{Scale=MatchLowercase}
  \defaultfontfeatures[\rmfamily]{Ligatures=TeX,Scale=1}
\fi
\usepackage{lmodern}
\ifPDFTeX\else  
    % xetex/luatex font selection
\fi
% Use upquote if available, for straight quotes in verbatim environments
\IfFileExists{upquote.sty}{\usepackage{upquote}}{}
\IfFileExists{microtype.sty}{% use microtype if available
  \usepackage[]{microtype}
  \UseMicrotypeSet[protrusion]{basicmath} % disable protrusion for tt fonts
}{}
\makeatletter
\@ifundefined{KOMAClassName}{% if non-KOMA class
  \IfFileExists{parskip.sty}{%
    \usepackage{parskip}
  }{% else
    \setlength{\parindent}{0pt}
    \setlength{\parskip}{6pt plus 2pt minus 1pt}}
}{% if KOMA class
  \KOMAoptions{parskip=half}}
\makeatother
\usepackage{xcolor}
\setlength{\emergencystretch}{3em} % prevent overfull lines
\setcounter{secnumdepth}{-\maxdimen} % remove section numbering
% Make \paragraph and \subparagraph free-standing
\ifx\paragraph\undefined\else
  \let\oldparagraph\paragraph
  \renewcommand{\paragraph}[1]{\oldparagraph{#1}\mbox{}}
\fi
\ifx\subparagraph\undefined\else
  \let\oldsubparagraph\subparagraph
  \renewcommand{\subparagraph}[1]{\oldsubparagraph{#1}\mbox{}}
\fi


\providecommand{\tightlist}{%
  \setlength{\itemsep}{0pt}\setlength{\parskip}{0pt}}\usepackage{longtable,booktabs,array}
\usepackage{calc} % for calculating minipage widths
% Correct order of tables after \paragraph or \subparagraph
\usepackage{etoolbox}
\makeatletter
\patchcmd\longtable{\par}{\if@noskipsec\mbox{}\fi\par}{}{}
\makeatother
% Allow footnotes in longtable head/foot
\IfFileExists{footnotehyper.sty}{\usepackage{footnotehyper}}{\usepackage{footnote}}
\makesavenoteenv{longtable}
\usepackage{graphicx}
\makeatletter
\def\maxwidth{\ifdim\Gin@nat@width>\linewidth\linewidth\else\Gin@nat@width\fi}
\def\maxheight{\ifdim\Gin@nat@height>\textheight\textheight\else\Gin@nat@height\fi}
\makeatother
% Scale images if necessary, so that they will not overflow the page
% margins by default, and it is still possible to overwrite the defaults
% using explicit options in \includegraphics[width, height, ...]{}
\setkeys{Gin}{width=\maxwidth,height=\maxheight,keepaspectratio}
% Set default figure placement to htbp
\makeatletter
\def\fps@figure{htbp}
\makeatother
% definitions for citeproc citations
\NewDocumentCommand\citeproctext{}{}
\NewDocumentCommand\citeproc{mm}{%
  \begingroup\def\citeproctext{#2}\cite{#1}\endgroup}
\makeatletter
 % allow citations to break across lines
 \let\@cite@ofmt\@firstofone
 % avoid brackets around text for \cite:
 \def\@biblabel#1{}
 \def\@cite#1#2{{#1\if@tempswa , #2\fi}}
\makeatother
\newlength{\cslhangindent}
\setlength{\cslhangindent}{1.5em}
\newlength{\csllabelwidth}
\setlength{\csllabelwidth}{3em}
\newenvironment{CSLReferences}[2] % #1 hanging-indent, #2 entry-spacing
 {\begin{list}{}{%
  \setlength{\itemindent}{0pt}
  \setlength{\leftmargin}{0pt}
  \setlength{\parsep}{0pt}
  % turn on hanging indent if param 1 is 1
  \ifodd #1
   \setlength{\leftmargin}{\cslhangindent}
   \setlength{\itemindent}{-1\cslhangindent}
  \fi
  % set entry spacing
  \setlength{\itemsep}{#2\baselineskip}}}
 {\end{list}}
\usepackage{calc}
\newcommand{\CSLBlock}[1]{\hfill\break\parbox[t]{\linewidth}{\strut\ignorespaces#1\strut}}
\newcommand{\CSLLeftMargin}[1]{\parbox[t]{\csllabelwidth}{\strut#1\strut}}
\newcommand{\CSLRightInline}[1]{\parbox[t]{\linewidth - \csllabelwidth}{\strut#1\strut}}
\newcommand{\CSLIndent}[1]{\hspace{\cslhangindent}#1}

\usepackage{booktabs}
\usepackage{longtable}
\usepackage{array}
\usepackage{multirow}
\usepackage{wrapfig}
\usepackage{float}
\usepackage{colortbl}
\usepackage{pdflscape}
\usepackage{tabu}
\usepackage{threeparttable}
\usepackage{threeparttablex}
\usepackage[normalem]{ulem}
\usepackage{makecell}
\usepackage{xcolor}
\usepackage{lineno}\linenumbers
\usepackage{amsmath}
\usepackage{graphicx,psfrag,epsf}
\usepackage{enumerate}
\usepackage{url}
\addtolength{\oddsidemargin}{-.5in}%
\addtolength{\evensidemargin}{-.5in}%
\addtolength{\textwidth}{1in}%
\addtolength{\textheight}{1.3in}%
\addtolength{\topmargin}{-.8in}%
\usepackage[noblocks]{authblk}
\renewcommand*{\Authsep}{, }
\renewcommand*{\Authand}{, }
\renewcommand*{\Authands}{, }
\renewcommand\Affilfont{\small}
\makeatletter
\@ifpackageloaded{caption}{}{\usepackage{caption}}
\AtBeginDocument{%
\ifdefined\contentsname
  \renewcommand*\contentsname{Table of contents}
\else
  \newcommand\contentsname{Table of contents}
\fi
\ifdefined\listfigurename
  \renewcommand*\listfigurename{List of Figures}
\else
  \newcommand\listfigurename{List of Figures}
\fi
\ifdefined\listtablename
  \renewcommand*\listtablename{List of Tables}
\else
  \newcommand\listtablename{List of Tables}
\fi
\ifdefined\figurename
  \renewcommand*\figurename{Figure}
\else
  \newcommand\figurename{Figure}
\fi
\ifdefined\tablename
  \renewcommand*\tablename{Table}
\else
  \newcommand\tablename{Table}
\fi
}
\@ifpackageloaded{float}{}{\usepackage{float}}
\floatstyle{ruled}
\@ifundefined{c@chapter}{\newfloat{codelisting}{h}{lop}}{\newfloat{codelisting}{h}{lop}[chapter]}
\floatname{codelisting}{Listing}
\newcommand*\listoflistings{\listof{codelisting}{List of Listings}}
\makeatother
\makeatletter
\makeatother
\makeatletter
\@ifpackageloaded{caption}{}{\usepackage{caption}}
\@ifpackageloaded{subcaption}{}{\usepackage{subcaption}}
\makeatother
\ifLuaTeX
  \usepackage{selnolig}  % disable illegal ligatures
\fi
\usepackage{bookmark}

\IfFileExists{xurl.sty}{\usepackage{xurl}}{} % add URL line breaks if available
\urlstyle{same} % disable monospaced font for URLs
\hypersetup{
  colorlinks=true,
  linkcolor={blue},
  filecolor={Maroon},
  citecolor={Blue},
  urlcolor={Blue},
  pdfcreator={LaTeX via pandoc}}




\date{}
\begin{document}

\setstretch{2}
\section{Title: Experiential learning with 3D data visualizations in an
introductory statistics
course}\label{title-experiential-learning-with-3d-data-visualizations-in-an-introductory-statistics-course}

\section{Abstract}\label{abstract}

A key component of statistics courses is to teach students how to
interpret data visualizations. Although many research-based
recommendations exist for creating graphs, the technological advances
for creating such graphs have outpaced studies that evaluate their
effectiveness to the status quo, especially with 3D graphs. Here, we
describe a process of integrating an experiment on 3D graphs as a
project for students enrolled in an introductory statistics course and
gather responses as students reflect on their positions as both
experiment participants and reviewers of empirical evidence. A total of
82 students participated in our graphics project and displayed a pattern
of not fully grasping research objectives as experiment participants; as
students reviewed material from our pilot study of the same design, they
tended to gain a clearer understanding about the purpose of the
experiment and its role in the realm of data visualizations. The project
we presented to students shows promise as an educational tool for
helping students gain a more holistic view of statistical research.

\section{Introduction}\label{introduction}

Students in introductory statistics courses are mainly exposed to
elementary methods and textbook examples of their applications. This is
in part due to the field's emphasis on teaching students to think
statistically using real data (\citeproc{ref-carver}{Carver, College,
and Everson 2016}). Many textbooks, such as Tintle et al.
(\citeproc{ref-Tintle2021}{2021}), take a single scenario and ask
students to perform its corresponding inferential test. This process is
then repeated over the course content without much deviation.

In some cases, students have the opportunity to participate in
well-designed experiments in the classroom
(\citeproc{ref-mcgowan2011}{McGowan 2011}). This can expose students to
concepts such as randomization and let students see the specifics of
experimental design through their participation. Loy
(\citeproc{ref-loy2021a}{2021}) demonstrated that student participants
often recalled their experiment in later concepts, showing some evidence
that students can benefit from the hands-on experience.

One key aspect in the statistics classroom is teaching students to
interpret data through visualizations. Nearly 40 years ago, Cleveland
and McGill (\citeproc{ref-cleveland1984}{1984}) began the process of
establishing good practices for making graphs. While Cleveland and
McGill's findings have been replicated (\citeproc{ref-heer2010}{Heer and
Bostock 2010}), there are many areas in data visualization that remain
underdeveloped, such as 3D graphs. The current mantra is to avoid 3D
graphs when possible and studies around the 1990s seem to provide some
valid skepticism of their use. Barfield and Robless
(\citeproc{ref-barfield1989}{1989}) showed that 3D graphs were sometimes
better than 2D graphs depending on the participant's experience level,
but that participants were most confident with their answers for 2D
graphs. Fisher, Dempsey, and Marousky (\citeproc{ref-fisher1997}{1997})
also observed a preference for 2D graphs over 3D graphs when extracting
information while simultaneously showing no preference for visual appeal
for either graph type. A major limitation of these studies is that the
3D graphs were 2D projections and not ``true'' 3D graphs. This is
somewhat addressed by Kraus et al. (\citeproc{ref-kraus2020}{2020}) with
the use of virtual reality, but effectively rendering ``true'' 3D graphs
is largely unexplored.

This underdeveloped area of 3D graphs provides a unique opportunity to
be used as an experiential learning opportunity for statistics students.
Not only can students benefit from the exposure to different graph
types, but they can also see how research is conducted through the lens
of a participant and researcher. While it is unclear how students will
respond to active research as a teaching method, it may be beneficial
for reinforcing statistical thinking.

In this paper, we discuss the use of an experiential learning module in
an introductory statistics classroom environment and its potential
application as an educational tool.

\section{Methods}\label{methods}

We introduced students enrolled in STAT 218, the introductory statistics
course at University of Nebraska-Lincoln, to a graphics project that
contained an experiment and progressively revealing components that
illustrate the experiment's research objectives. The project started by
providing students with minimal information about the research
objectives before revealing the scope of the experiment through an
extended abstract and presentation. The goal of the graphics project is
to observe how students think statistically about experiments from the
viewpoint of participants and researchers.

\subsection{Experiential Learning}\label{experiential-learning}

The classroom integration of the graphics experiment project was split
into two stages: research participation and reflection of the overall
research objectives. In the research participation stage, students
participated in the experiment with the understanding that the
experiment is testing for how people perceive statistical graphics.
Students were not informed of the specific research hypotheses when
participating in the experiment. After participating in the experiment,
students were provided materials that cover the research objectives and
reflected on their new understanding of the experiment's purpose.

Within the research participation stage, students completed four
modules: informed consent, pre-experiment reflection, experiment
participation, and post-experiment reflection. The informed consent
asked students if they consent to their responses being shared with the
researchers and if they are 19 years of age or older. Within the
informed consent module, students were informed that their data would be
anonymized and that the experiment was carried out in accordance to the
institutional review board (Project ID: 22579). In the pre-experiment
reflection, students were asked to write a paragraph about how they
think the process of scientific investigation looks from the perspective
of researchers and the general public. The experiment participation
module asked students to paste the code generated from the experiment,
which is detailed in the next section. For the post-experiment
reflection, students were asked five questions about the purpose of the
experiment. These include questions on the hypotheses being tested,
sources of error, variables of interest, and elements of experimental
design.

After completing the experiment reflections, students moved to the
reflection of the overall research objectives. Students were first
directed to read a two-page extended abstract that we submitted as a
contributed paper for the Symposium on Data Science \& Statistics (SDSS)
conference in 2023. The extended abstract outlined the experiment's
purpose and procedures, but not the results from our initial pilot
study. After reading the extended abstract, students were asked to write
a paragraph about what they found clearer about the experiment's purpose
than when they were a participant. Finally, students were directed to
watch a 12-minute pre-recorded presentation based on an abbreviated
version given at SDSS. The video contains the same material as the
extended abstract and included the results from our pilot study. The
presentation reflection asked students four questions about the
experiment and how information was presented differently than in the
extended abstract.

Except for the informed consent and experiment participation modules,
all student responses were open-ended. Each question and its
corresponding module can be found in Table 1.

Instructors for STAT 218 were recruited for Summer 2023 and Fall 2023 to
administer the graphics project into their classroom. The instructors
were given the option of administering the project as coursework
material or extra credit, along with the liberty of grading at their own
discretion. While all students were given the option to participate in
the graphics project, we were only able to collect responses when the
informed consent was obtained and if the student was 19 years of age.

\subsection{Graphics Experiment}\label{graphics-experiment}

We took inspiration from Cleveland and McGill's seminal work
(\citeproc{ref-cleveland1984}{1984}) on graphical perception to design
our graphics study. Participants were presented with a series of bar
graphs where two bars are marked with either a circle or triangle. The
heights of the bars were chosen from the following equation:

\begin{equation}\phantomsection\label{eq-vals}{s_i=10\cdot 10^{(i-1)/12}, \qquad i=1,...,10}\end{equation}

Values from Equation~\ref{eq-vals} were then paired such that the ratio
of the smaller value to the larger value yield the ratios of 17.8, 26.1,
38.3, 46.4, 56.2, 68.1, and 82.5. Each bar graph has two groupings of
five bars. The value pairs for each ratio were either placed in the
first grouping on the second and third bars (adjacent), or placed on the
second bars in each grouping (separated). This follows the Type 1 and
Type 3 graphs from Cleveland and McGill's position-length experiment,
respectively.

Deviating from Cleveland and McGill, we introduced four plot types: 2D
digital, 3D digital (static), 3D digital (interactive), and 3D printed.
There was no single software package that could create all four plot
types, so we carefully constructed graphs from different software
packages to be as similar as possible. The 2D digital plots were
rendered with the \texttt{ggplot2} package
(\citeproc{ref-ggplot2}{Wickham 2016}). Microsoft Excel was used to
render the 3D digital (static) plots (\citeproc{ref-msexcel}{Microsoft
Corporation 2018}). The 3D digital (interactive) and 3D printed plots
were created with OpenSCAD
(\citeproc{ref-kintelOpenSCADDocumentation2023}{Kintel 2023}), where the
generated STL files for the 3D digital (interactive) plots were rendered
with the \texttt{rgl} package (\citeproc{ref-rgl}{Murdoch and Adler
2023}).

With 56 treatment combinations, we opted to use an incomplete block
design to provide participants with 15-20 graphs. Kits of graphs were
constructed so that five of the seven ratios are equally represented,
resulting in 21 different kits. Within each kit, all graph types
appeared for each ratio and the comparison type was randomly assigned. A
visual layout of the experiment is shown in Figure 1. All kits received
a unique identifier and a set of instructions for accessing the
experiment.

A Shiny application (\citeproc{ref-shiny}{Chang et al. 2023}) was
designed to administer the experiment. Students were directed to
randomly select a kit of graphs and visit the Shiny application's url
linked on the instructions. For students enrolled in the online sections
of STAT 218, the url was provided by the instructor and they were
prompted in the application to select that they were an online
participant. After students provided a kit identifier (if applicable),
students were presented graphs in a randomized order. If the student
marked that they were an online participant, the 3D printed graphs were
removed from their experiment lineup. Each graph asked the students to
first identify the larger marked bar and then to guess the height of the
smaller marked bar if the larger marked bar was 100 units tall using a
slider widget. After completing the experiment, students were provided
with a code to copy and paste into the experiment participation module.

\subsection{Data Analysis}\label{data-analysis}

Since nearly all of the student responses to the project modules are
open-ended, the analysis of the project is qualitative in nature. We
will extract selected responses that we feel highlight common themes
among the students or other points of interest. For paragraph responses,
bigrams are used to illustrate word pairs after removing stop words
(e.g., ``the'' and ``and''). This will help in establishing common
themes that appear in student responses. We opt to reserve the results
of the graphics experiment for another paper so that we can clearly
differentiate between our findings in graphics project and the
experiment's role in a classroom environment.

\section{Results}\label{results}

Given the nature of the recruitment method, we were only able to recruit
3 instructors for summer and fall semesters in 2023. Each instructor
offered the project as extra credit and student participation was
entirely voluntary. A total of 82 students participated in the project;
a summary of student participation is presented in Table 1. There were 9
students who did not complete the project in its entirety.

\subsection{Selected Responses from
Reflections}\label{selected-responses-from-reflections}

Prior to the experiment, students generally understood the purpose
scientific research by connecting the ideas of hypothesis testing and
publishing results. Students wrote about scientific research starting
from the place of a question, followed by conducting an experiment and
relaying the results to the public. A bigram plot from the
Pre-Experiment Reflection is shown in Figure 3, which highlights the
recurring trends and patterns in the student paragraph responses.

Some students correctly identified parts of the questions asked in the
post-experiment reflection, but many missed key components.

\textbf{What do you think the purpose of the experiment was?}

\begin{itemize}
\item
  ``They could be trying to determine how different genders, ages, etc.
  perceive the sizes of the bars in the graph. Demographics could make a
  pretty significant difference.''
\item
  ``I think the purpose of this experiment was for the researcher to
  gather data on how people perceive, interpret, and understand 3D
  graphs.''
\item
  ``I think this experiment aimed to test if it was easier to compare
  two graphs in 2D or 3D.''
\item
  ``To gage students skills at estimating relative size ratios.''
\end{itemize}

\textbf{What hypotheses might the experimenters have been testing?}

\begin{itemize}
\item
  ``Do students change their answers when asked the same question over
  and over?''
\item
  ``How taking Statistics 218 effects how you can compare two groups.''
\item
  ``That 2d is preferred over 3d. It cleans up the data presentation.''
\item
  ``They might have been testing if a 2D model is easier to estimate its
  relative size to another when compared to a 3D model of it.''
\end{itemize}

\textbf{What sources of error are involved in this experiment?}

\begin{itemize}
\item
  ``Misunderstanding of task, technical issues''
\item
  ``If people are just randomly picking answers.''
\item
  ``Fatigue effect over the course of making many judgements, learning
  patterns from seeing the same ratios multiple times, possibly
  difference in eyesight among participants.''
\item
  ``As far as I know, there wasn't much random sampling involved or
  there may be some bias of sorts. The results may apply for students in
  STATS 218 here at UNL, but maybe not for other students taking a
  similar statistics class elsewhere.''
\end{itemize}

\textbf{What elements of experimental design, such as randomization or
the use of a control group, do you think were present in the experiment?
Why?}

\begin{itemize}
\item
  ``random students in the stats class''
\item
  ``Randomization: The survey used an experimental design where in the
  survey there were different sets of 3D charts and maybe by a
  randomization process each participant was shown a different set of
  charts to see the differences in interpretations of the charts based
  on which set was assigned. Control Group: Since this survey aimed to
  only understand how participants interpret 3D charts without
  comparison to other chart types, then no control group was needed.''
\item
  ``Randomization was not used because it was offered as an extra credit
  assignment in class.''
\item
  ``Randomization was used because the ever person got a different
  graph.''
\end{itemize}

The abstract unveiled the scope of the study to students, many of whom
did not realize the underlying complexities. Nearly all students
responded with statements about gaining clarity about the purpose the
experiment and its role in testing the differences between 2D and 3D
graphs. A bigram plot of the student responses to the abstract
reflection prompt is shown in Figure 4.

Lastly, more than half of the students (78.5\%) responded that they
preferred the presentation over the extended abstract.

\begin{itemize}
\item
  ``I am a visual learner so I would have rather heard about in through
  the presentation. It also broke down the steps which is easier for me
  to understand. I think the presentation as a whole would be better for
  determining how the experiment is designed.''
\item
  ``I would prefer the presentation because it gives the audience more
  information about the experiment rather than the extended abstract.
  The presentation goes over the results of the experiment and explains
  what they mean using graphs and other visuals.{[}\ldots{]}''
\item
  ``Personally I like the abstract better. If I get confused on
  something it is so much easier to go back and reread to understand
  what is going on. If I ask myself questions about it, it is much
  easier to go back and find answers to the questions as well.''
\end{itemize}

\section{Discussion}\label{discussion}

Our goal was to provide students of an introductory statistics course
the opportunity to reflect on active research. Students generally
appreciated the progressively revealing nature of the graphics project,
which is evident from the abstract and presentation reflections. When
provided with the post-experiment reflections, students often either
missed the research objective of the experiment or had partially correct
responses. The abstract reflection received many responses indicating
that students had moments of realization about the true nature of our
research goals, which was further expanded in the presentation
reflection prompts. The reflections indicated that students were
thoughtful, and sometimes amusing, with their responses and that they
were on the path of statistical thinking.

A limitation of this study is the use of open-ended responses that do
not assess student learning. While the student responses were useful in
gathering insight, the responses are widely varied and do not have
direct comparisons of statistical thinking throughout the modules.
Another limiting factor is tiered layering of convenience sampling, with
instructors being recruited first before recruiting students. This makes
it impossible to generalize our results to statistics students, let
alone the students at University of Nebraska-Lincoln.

In future studies, we plan to use a similar framework to conduct
experiments on more typical 3-dimensional data, such as heatmaps. The
use of graphical experiments in the classroom not only provides a
readily available convenience sample, but also adheres to the
recommendations of the Guidelines for Assessment and Instruction in
Statistics Education (GAISE) College Report
(\citeproc{ref-carver}{Carver, College, and Everson 2016}). With the
framework we provided in this paper, we aim to make adjustments to
further improve the graphics experiment and corresponding project as an
experiential learning opportunity.

\section{References}\label{references}

Link to journal citation style:
\href{https://www.tandfonline.com/action/authorSubmission?show=instructions&journalCode=ujse21\#refs}{here}

\phantomsection\label{refs}
\begin{CSLReferences}{1}{0}
\bibitem[\citeproctext]{ref-barfield1989}
Barfield, Woodrow, and Robert Robless. 1989. {``The Effects of Two- or
Three-Dimensional Graphics on the Problem-Solving Performance of
Experienced and Novice Decision Makers.''} \emph{Behaviour \&
Information Technology} 8 (5): 369--85.
\url{https://doi.org/10.1080/01449298908914567}.

\bibitem[\citeproctext]{ref-carver}
Carver, Robert, Stonehill College, and Michelle Everson. 2016.
{``Guidelines for Assessment and Instruction in Statistics Education
(GAISE) College Report.''}

\bibitem[\citeproctext]{ref-shiny}
Chang, Winston, Joe Cheng, JJ Allaire, Carson Sievert, Barret Schloerke,
Yihui Xie, Jeff Allen, Jonathan McPherson, Alan Dipert, and Barbara
Borges. 2023. {``Shiny: Web Application Framework for r.''}

\bibitem[\citeproctext]{ref-cleveland1984}
Cleveland, William S., and Robert McGill. 1984. {``Graphical Perception:
Theory, Experimentation, and Application to the Development of Graphical
Methods.''} \emph{Journal of the American Statistical Association} 79
(387): 531--54. \url{https://doi.org/10.1080/01621459.1984.10478080}.

\bibitem[\citeproctext]{ref-fisher1997}
Fisher, Samuel H., John V. Dempsey, and Robert T. Marousky. 1997.
{``Data Visualization: Preference and Use of Two-Dimensional and
Three-Dimensional Graphs.''} \emph{Social Science Computer Review} 15
(3): 256--63. \url{https://doi.org/10.1177/089443939701500303}.

\bibitem[\citeproctext]{ref-heer2010}
Heer, Jeffrey, and Michael Bostock. 2010. {``CHI '10: CHI Conference on
Human Factors in Computing Systems.''} In, 203--12. Atlanta Georgia USA:
ACM. \url{https://doi.org/10.1145/1753326.1753357}.

\bibitem[\citeproctext]{ref-kintelOpenSCADDocumentation2023}
Kintel, Marius. 2023. {``{OpenSCAD}. {OpenSCAD}.org.''} July 17, 2023.

\bibitem[\citeproctext]{ref-kraus2020}
Kraus, Matthias, Katrin Angerbauer, Juri Buchmüller, Daniel Schweitzer,
Daniel A. Keim, Michael Sedlmair, and Johannes Fuchs. 2020. {``CHI '20:
CHI Conference on Human Factors in Computing Systems.''} In, 1--14.
Honolulu HI USA: ACM. \url{https://doi.org/10.1145/3313831.3376675}.

\bibitem[\citeproctext]{ref-loy2021a}
Loy, Adam. 2021. {``Bringing Visual Inference to the Classroom.''}
\emph{Journal of Statistics and Data Science Education} 29 (2): 171--82.
\url{https://doi.org/10.1080/26939169.2021.1920866}.

\bibitem[\citeproctext]{ref-mcgowan2011}
McGowan, Herle M. 2011. {``Planning a Comparative Experiment in
Educational Settings.''} \emph{Journal of Statistics Education} 19 (2):
4. \url{https://doi.org/10.1080/10691898.2011.11889612}.

\bibitem[\citeproctext]{ref-msexcel}
Microsoft Corporation. 2018. {``Microsoft Excel.''}

\bibitem[\citeproctext]{ref-rgl}
Murdoch, Duncan, and Daniel Adler. 2023. {``Rgl: 3D Visualization Using
OpenGL.''}

\bibitem[\citeproctext]{ref-Tintle2021}
Tintle, Nathan, Beth L Chance, George W Cobb, Allan J Rossman, Soma Roy,
Todd Swanson, and Jill VanderStoep. 2021. \emph{Introduction to
Statistical Investigations}.

\bibitem[\citeproctext]{ref-ggplot2}
Wickham, Hadley. 2016. {``Ggplot2: Elegant Graphics for Data
Analysis.''}

\end{CSLReferences}

\section{Figures}\label{figures}

Figure 1: Visual display of the experimental design for students who
participated in the 3D bar charts experiment. Kits of graphs were
created by first choosing five ratios from nine available options (1).
Each ratio then uses all graph types, with the exception of the 3D
printed graphs for online students (2). Finally, all graphs were
randomly assigned to have the marked bars as adjacent or separated (3).

Figure 2: Bigram of student responses to the pre-experiment prompt. Each
line represents pairs of words that appeared together where each pair
occurred at least twice. Students generally understood that science is
about investigating research questions and collecting data.

Figure 3: Bigram of student responses to the abstract reflection prompt.
Each line represents pairs of words that appeared together where each
pair occurred at least twice. Students generally understood that science
is about investigating research questions and collecting data.

\section{Tables}\label{tables}

\begin{table}

\caption{\label{tab:unnamed-chunk-4}Questions provided to students in each project module.}
\centering
\fontsize{10}{12}\selectfont
\begin{tabu} to \linewidth {>{\raggedright\arraybackslash}p{9em}>{\raggedright\arraybackslash}p{6em}>{\raggedright\arraybackslash}p{25em}}
\toprule
Reflection & Question & Prompt\\
\midrule
Pre-Experiment & Q3 & In this class, you'll be learning about the process of scientific investigation. What do you think that process looks like, from the perspective of a researcher, compared to what it looks like from the perspective of someone in the general public who is a consumer of scientific results? Write a paragraph (at least 3-5 sentences) about how you think science happens.\\
\cmidrule{1-3}
 & Q5 & What do you think the purpose of the experiment was?\\
\cmidrule{2-3}
 & Q6 & What hypotheses might the experimenter have been testing?\\
\cmidrule{2-3}
 & Q7 & What sources of error are involved in this experiment?\\
\cmidrule{2-3}
 & Q8 & What variables were examined? For each variable, identify whether it was quantitative or categorical.\\
\cmidrule{2-3}
\multirow{-5}{*}{\raggedright\arraybackslash Post-Experiment} & Q9 & What elements of experimental design, such as randomization or the use of a control group, do you think were present in the experiment? Why?\\
\cmidrule{1-3}
Abstract & Q10 & What components of the experiment are clearer now than they were as a participant? What questions do you still have for the experimenter? Write 3-5 sentences reflecting on the abstract.\\
\cmidrule{1-3}
 & Q11 & How did the information you gained from the components of this project (participation, post-study reflection, extended abstract, presentation) differ?\\
\cmidrule{2-3}
 & Q12 & What components were emphasized in the presentation that weren't emphasized in the abstract? Why do you think that is?\\
\cmidrule{2-3}
 & Q13 & What critiques do you have of this study and its design? What would have made the study better?\\
\cmidrule{2-3}
\multirow{-4}{*}{\raggedright\arraybackslash Presentation} & Q14 & If you had to hear about this study using only the extended abstract or only the presentation, which one would you prefer? Which one would be better for determining whether the experiment was well designed?\\
\bottomrule
\end{tabu}
\end{table}

\begin{table}[H]

\begin{threeparttable}
\caption{\label{tab:unnamed-chunk-5}Number of valid student participants by semester.}
\centering
\begin{tabular}[t]{lrr}
\toprule
Semester & Number of Sections & Number of Students\\
\midrule
Summer 2023 (May-June) & 1 & 17\\
Summer 2023 (July-Aug) & 1 & 23\\
Fall 2023 (May-June) & 1 & 42\\
\bottomrule
\end{tabular}
\begin{tablenotes}
\small
\item [] Students under 19 years of age or did not consent were exluded from data collection. To comply with IRB, no demographic information was collected in order to keep students anonymous.
\end{tablenotes}
\end{threeparttable}
\end{table}



\end{document}
