% Options for packages loaded elsewhere
%DIF LATEXDIFF DIFFERENCE FILE
%DIF DEL manuscript-V2.tex   Tue Apr 23 21:15:32 2024
%DIF ADD manuscript-V3.tex   Mon May  6 22:40:30 2024
\PassOptionsToPackage{unicode}{hyperref}
\PassOptionsToPackage{hyphens}{url}
\PassOptionsToPackage{dvipsnames,svgnames,x11names}{xcolor}
%
\documentclass[
  12pt,
]{article}

\usepackage{amsmath,amssymb}
\usepackage{setspace}
\usepackage{iftex}
\ifPDFTeX
  \usepackage[T1]{fontenc}
  \usepackage[utf8]{inputenc}
  \usepackage{textcomp} % provide euro and other symbols
\else % if luatex or xetex
  \usepackage{unicode-math}
  \defaultfontfeatures{Scale=MatchLowercase}
  \defaultfontfeatures[\rmfamily]{Ligatures=TeX,Scale=1}
\fi
\usepackage{lmodern}
\ifPDFTeX\else  
    % xetex/luatex font selection
\fi
% Use upquote if available, for straight quotes in verbatim environments
\IfFileExists{upquote.sty}{\usepackage{upquote}}{}
\IfFileExists{microtype.sty}{% use microtype if available
  \usepackage[]{microtype}
  \UseMicrotypeSet[protrusion]{basicmath} % disable protrusion for tt fonts
}{}
\makeatletter
\@ifundefined{KOMAClassName}{% if non-KOMA class
  \IfFileExists{parskip.sty}{%
    \usepackage{parskip}
  }{% else
    \setlength{\parindent}{0pt}
    \setlength{\parskip}{6pt plus 2pt minus 1pt}}
}{% if KOMA class
  \KOMAoptions{parskip=half}}
\makeatother
\usepackage{xcolor}
\setlength{\emergencystretch}{3em} % prevent overfull lines
\setcounter{secnumdepth}{-\maxdimen} % remove section numbering
% Make \paragraph and \subparagraph free-standing
\ifx\paragraph\undefined\else
  \let\oldparagraph\paragraph
  \renewcommand{\paragraph}[1]{\oldparagraph{#1}\mbox{}}
\fi
\ifx\subparagraph\undefined\else
  \let\oldsubparagraph\subparagraph
  \renewcommand{\subparagraph}[1]{\oldsubparagraph{#1}\mbox{}}
\fi


\providecommand{\tightlist}{%
  \setlength{\itemsep}{0pt}\setlength{\parskip}{0pt}}\usepackage{longtable,booktabs,array}
\usepackage{calc} % for calculating minipage widths
% Correct order of tables after \paragraph or \subparagraph
\usepackage{etoolbox}
\makeatletter
\patchcmd\longtable{\par}{\if@noskipsec\mbox{}\fi\par}{}{}
\makeatother
% Allow footnotes in longtable head/foot
\IfFileExists{footnotehyper.sty}{\usepackage{footnotehyper}}{\usepackage{footnote}}
\makesavenoteenv{longtable}
\usepackage{graphicx}
\makeatletter
\def\maxwidth{\ifdim\Gin@nat@width>\linewidth\linewidth\else\Gin@nat@width\fi}
\def\maxheight{\ifdim\Gin@nat@height>\textheight\textheight\else\Gin@nat@height\fi}
\makeatother
% Scale images if necessary, so that they will not overflow the page
% margins by default, and it is still possible to overwrite the defaults
% using explicit options in \includegraphics[width, height, ...]{}
\setkeys{Gin}{width=\maxwidth,height=\maxheight,keepaspectratio}
% Set default figure placement to htbp
\makeatletter
\def\fps@figure{htbp}
\makeatother
% definitions for citeproc citations
\NewDocumentCommand\citeproctext{}{}
\NewDocumentCommand\citeproc{mm}{%
  \begingroup\def\citeproctext{#2}\cite{#1}\endgroup}
\makeatletter
 % allow citations to break across lines
 \let\@cite@ofmt\@firstofone
 % avoid brackets around text for \cite:
 \def\@biblabel#1{}
 \def\@cite#1#2{{#1\if@tempswa , #2\fi}}
\makeatother
\newlength{\cslhangindent}
\setlength{\cslhangindent}{1.5em}
\newlength{\csllabelwidth}
\setlength{\csllabelwidth}{3em}
\newenvironment{CSLReferences}[2] % #1 hanging-indent, #2 entry-spacing
 {\begin{list}{}{%
  \setlength{\itemindent}{0pt}
  \setlength{\leftmargin}{0pt}
  \setlength{\parsep}{0pt}
  % turn on hanging indent if param 1 is 1
  \ifodd #1
   \setlength{\leftmargin}{\cslhangindent}
   \setlength{\itemindent}{-1\cslhangindent}
  \fi
  % set entry spacing
  \setlength{\itemsep}{#2\baselineskip}}}
 {\end{list}}
\usepackage{calc}
\newcommand{\CSLBlock}[1]{\hfill\break\parbox[t]{\linewidth}{\strut\ignorespaces#1\strut}}
\newcommand{\CSLLeftMargin}[1]{\parbox[t]{\csllabelwidth}{\strut#1\strut}}
\newcommand{\CSLRightInline}[1]{\parbox[t]{\linewidth - \csllabelwidth}{\strut#1\strut}}
\newcommand{\CSLIndent}[1]{\hspace{\cslhangindent}#1}

\usepackage{booktabs}
\usepackage{longtable}
\usepackage{array}
\usepackage{multirow}
\usepackage{wrapfig}
\usepackage{float}
\usepackage{colortbl}
\usepackage{pdflscape}
\usepackage{tabu}
\usepackage{threeparttable}
\usepackage{threeparttablex}
\usepackage[normalem]{ulem}
\usepackage{makecell}
\usepackage{xcolor}
\usepackage{lineno}\linenumbers
\usepackage{amsmath}
\usepackage{graphicx,psfrag,epsf}
\usepackage{enumerate}
\usepackage{url}
\addtolength{\oddsidemargin}{-.5in}%
\addtolength{\evensidemargin}{-.5in}%
\addtolength{\textwidth}{1in}%
\addtolength{\textheight}{1.3in}%
\addtolength{\topmargin}{-.8in}%
\usepackage[noblocks]{authblk}
\renewcommand*{\Authsep}{, }
\renewcommand*{\Authand}{, }
%DIF 141a141
\providecommand{\DIFdel}[1]{} %DIF > 
%DIF -------
\renewcommand*{\Authands}{, }
\renewcommand\Affilfont{\small}
\makeatletter
\@ifpackageloaded{caption}{}{\usepackage{caption}}
\AtBeginDocument{%
\ifdefined\contentsname
  \renewcommand*\contentsname{Table of contents}
\else
  \newcommand\contentsname{Table of contents}
\fi
\ifdefined\listfigurename
  \renewcommand*\listfigurename{List of Figures}
\else
  \newcommand\listfigurename{List of Figures}
\fi
\ifdefined\listtablename
  \renewcommand*\listtablename{List of Tables}
\else
  \newcommand\listtablename{List of Tables}
\fi
\ifdefined\figurename
  \renewcommand*\figurename{Figure}
\else
  \newcommand\figurename{Figure}
\fi
\ifdefined\tablename
  \renewcommand*\tablename{Table}
\else
  \newcommand\tablename{Table}
\fi
}
\@ifpackageloaded{float}{}{\usepackage{float}}
\floatstyle{ruled}
\@ifundefined{c@chapter}{\newfloat{codelisting}{h}{lop}}{\newfloat{codelisting}{h}{lop}[chapter]}
\floatname{codelisting}{Listing}
\newcommand*\listoflistings{\listof{codelisting}{List of Listings}}
\makeatother
\makeatletter
\makeatother
\makeatletter
\@ifpackageloaded{caption}{}{\usepackage{caption}}
\@ifpackageloaded{subcaption}{}{\usepackage{subcaption}}
\makeatother
\ifLuaTeX
  \usepackage{selnolig}  % disable illegal ligatures
\fi
\usepackage{bookmark}

\IfFileExists{xurl.sty}{\usepackage{xurl}}{} % add URL line breaks if available
\urlstyle{same} % disable monospaced font for URLs
\hypersetup{
  colorlinks=true,
  linkcolor={blue},
  filecolor={Maroon},
  citecolor={Blue},
  urlcolor={Blue},
  pdfcreator={LaTeX via pandoc}}




\date{}
%DIF PREAMBLE EXTENSION ADDED BY LATEXDIFF
%DIF CTRADITIONAL PREAMBLE %DIF PREAMBLE
\RequirePackage{color}\definecolor{RED}{rgb}{1,0,0}\definecolor{BLUE}{rgb}{0,0,1} %DIF PREAMBLE
\RequirePackage[stable]{footmisc} %DIF PREAMBLE
\DeclareOldFontCommand{\sf}{\normalfont\sffamily}{\mathsf} %DIF PREAMBLE
\providecommand{\DIFadd}[1]{{\protect\color{blue} \sf #1}} %DIF PREAMBLE
\providecommand{\DIFdel}[1]{{\protect\color{red} [..\footnote{removed: #1} ]}} %DIF PREAMBLE
%DIF SAFE PREAMBLE %DIF PREAMBLE
\providecommand{\DIFaddbegin}{} %DIF PREAMBLE
\providecommand{\DIFaddend}{} %DIF PREAMBLE
\providecommand{\DIFdelbegin}{} %DIF PREAMBLE
\providecommand{\DIFdelend}{} %DIF PREAMBLE
\providecommand{\DIFmodbegin}{} %DIF PREAMBLE
\providecommand{\DIFmodend}{} %DIF PREAMBLE
%DIF FLOATSAFE PREAMBLE %DIF PREAMBLE
\providecommand{\DIFaddFL}[1]{\DIFadd{#1}} %DIF PREAMBLE
\providecommand{\DIFdelFL}[1]{\DIFdel{#1}} %DIF PREAMBLE
\providecommand{\DIFaddbeginFL}{} %DIF PREAMBLE
\providecommand{\DIFaddendFL}{} %DIF PREAMBLE
\providecommand{\DIFdelbeginFL}{} %DIF PREAMBLE
\providecommand{\DIFdelendFL}{} %DIF PREAMBLE
\newcommand{\DIFscaledelfig}{0.5}
%DIF HIGHLIGHTGRAPHICS PREAMBLE %DIF PREAMBLE
\RequirePackage{settobox} %DIF PREAMBLE
\RequirePackage{letltxmacro} %DIF PREAMBLE
\newsavebox{\DIFdelgraphicsbox} %DIF PREAMBLE
\newlength{\DIFdelgraphicswidth} %DIF PREAMBLE
\newlength{\DIFdelgraphicsheight} %DIF PREAMBLE
% store original definition of \includegraphics %DIF PREAMBLE
\LetLtxMacro{\DIFOincludegraphics}{\includegraphics} %DIF PREAMBLE
\newcommand{\DIFaddincludegraphics}[2][]{{\color{blue}\fbox{\DIFOincludegraphics[#1]{#2}}}} %DIF PREAMBLE
\newcommand{\DIFdelincludegraphics}[2][]{% %DIF PREAMBLE
\sbox{\DIFdelgraphicsbox}{\DIFOincludegraphics[#1]{#2}}% %DIF PREAMBLE
\settoboxwidth{\DIFdelgraphicswidth}{\DIFdelgraphicsbox} %DIF PREAMBLE
\settoboxtotalheight{\DIFdelgraphicsheight}{\DIFdelgraphicsbox} %DIF PREAMBLE
\scalebox{\DIFscaledelfig}{% %DIF PREAMBLE
\parbox[b]{\DIFdelgraphicswidth}{\usebox{\DIFdelgraphicsbox}\\[-\baselineskip] \rule{\DIFdelgraphicswidth}{0em}}\llap{\resizebox{\DIFdelgraphicswidth}{\DIFdelgraphicsheight}{% %DIF PREAMBLE
\setlength{\unitlength}{\DIFdelgraphicswidth}% %DIF PREAMBLE
\begin{picture}(1,1)% %DIF PREAMBLE
\thicklines\linethickness{2pt} %DIF PREAMBLE
{\color[rgb]{1,0,0}\put(0,0){\framebox(1,1){}}}% %DIF PREAMBLE
{\color[rgb]{1,0,0}\put(0,0){\line( 1,1){1}}}% %DIF PREAMBLE
{\color[rgb]{1,0,0}\put(0,1){\line(1,-1){1}}}% %DIF PREAMBLE
\end{picture}% %DIF PREAMBLE
}\hspace*{3pt}}} %DIF PREAMBLE
} %DIF PREAMBLE
\LetLtxMacro{\DIFOaddbegin}{\DIFaddbegin} %DIF PREAMBLE
\LetLtxMacro{\DIFOaddend}{\DIFaddend} %DIF PREAMBLE
\LetLtxMacro{\DIFOdelbegin}{\DIFdelbegin} %DIF PREAMBLE
\LetLtxMacro{\DIFOdelend}{\DIFdelend} %DIF PREAMBLE
\DeclareRobustCommand{\DIFaddbegin}{\DIFOaddbegin \let\includegraphics\DIFaddincludegraphics} %DIF PREAMBLE
\DeclareRobustCommand{\DIFaddend}{\DIFOaddend \let\includegraphics\DIFOincludegraphics} %DIF PREAMBLE
\DeclareRobustCommand{\DIFdelbegin}{\DIFOdelbegin \let\includegraphics\DIFdelincludegraphics} %DIF PREAMBLE
\DeclareRobustCommand{\DIFdelend}{\DIFOaddend \let\includegraphics\DIFOincludegraphics} %DIF PREAMBLE
\LetLtxMacro{\DIFOaddbeginFL}{\DIFaddbeginFL} %DIF PREAMBLE
\LetLtxMacro{\DIFOaddendFL}{\DIFaddendFL} %DIF PREAMBLE
\LetLtxMacro{\DIFOdelbeginFL}{\DIFdelbeginFL} %DIF PREAMBLE
\LetLtxMacro{\DIFOdelendFL}{\DIFdelendFL} %DIF PREAMBLE
\DeclareRobustCommand{\DIFaddbeginFL}{\DIFOaddbeginFL \let\includegraphics\DIFaddincludegraphics} %DIF PREAMBLE
\DeclareRobustCommand{\DIFaddendFL}{\DIFOaddendFL \let\includegraphics\DIFOincludegraphics} %DIF PREAMBLE
\DeclareRobustCommand{\DIFdelbeginFL}{\DIFOdelbeginFL \let\includegraphics\DIFdelincludegraphics} %DIF PREAMBLE
\DeclareRobustCommand{\DIFdelendFL}{\DIFOaddendFL \let\includegraphics\DIFOincludegraphics} %DIF PREAMBLE
%DIF COLORLISTINGS PREAMBLE %DIF PREAMBLE
\RequirePackage{listings} %DIF PREAMBLE
\RequirePackage{color} %DIF PREAMBLE
\lstdefinelanguage{DIFcode}{ %DIF PREAMBLE
%DIF DIFCODE_CTRADITIONAL %DIF PREAMBLE
  moredelim=[il][\color{red}\scriptsize]{\%DIF\ <\ }, %DIF PREAMBLE
  moredelim=[il][\color{blue}\sffamily]{\%DIF\ >\ } %DIF PREAMBLE
} %DIF PREAMBLE
\lstdefinestyle{DIFverbatimstyle}{ %DIF PREAMBLE
	language=DIFcode, %DIF PREAMBLE
	basicstyle=\ttfamily, %DIF PREAMBLE
	columns=fullflexible, %DIF PREAMBLE
	keepspaces=true %DIF PREAMBLE
} %DIF PREAMBLE
\lstnewenvironment{DIFverbatim}{\lstset{style=DIFverbatimstyle}}{} %DIF PREAMBLE
\lstnewenvironment{DIFverbatim*}{\lstset{style=DIFverbatimstyle,showspaces=true}}{} %DIF PREAMBLE
%DIF END PREAMBLE EXTENSION ADDED BY LATEXDIFF

\begin{document}

\setstretch{2}
\DIFdelbegin \textbf{\DIFdel{Title:}} %DIFAUXCMD
\DIFdel{The use of a 3D graphics experiment as an experiential
learning opportunity in an introductory statistics course
}\DIFdelend \DIFaddbegin \section{\DIFadd{Title: Using a 3D graphics experiment for experiential learning
in an introductory statistics
course.}}\label{title-using-a-3d-graphics-experiment-for-experiential-learning-in-an-introductory-statistics-course.}
\DIFaddend 

\DIFdelbegin \textbf{\DIFdel{Running Title:}} %DIFAUXCMD
\DIFdel{Experiential learning with 3D graphs
}\DIFdelend \DIFaddbegin \section{\DIFadd{Running Title: Experiential learning with 3D
graphs}}\label{running-title-experiential-learning-with-3d-graphs}
\DIFaddend 

Tyler Wiederich\textsuperscript{1} (twiederich2@huskers.unl.edu)\DIFaddbegin \DIFadd{, Susan
VanderPlas\textsuperscript{1,*} (susan.vanderplas@unl.edu)
}\DIFaddend 

\DIFaddbegin \DIFadd{\textsuperscript{\emph{1}} }\DIFaddend \emph{\DIFdelbegin \DIFdel{\textsuperscript{1} }\DIFdelend Department of Statistics, University of
Nebraska-Lincoln, NE 68588}

\textsuperscript{*}\emph{To whom correspondence should be addressed}

Dr.~Susan VanderPlas, Department of Statistics, University of
Nebraska-Lincoln, 3310 Holdrege St, Lincoln, NE 68503, E-mail:
susan.vanderplas@unl.edu

\textbf{Keywords:} 3D graphs, statistics education

\section{Conflict of interest
statement}\label{conflict-of-interest-statement}

Susan VanderPlas and Tyler Wiederich do not declare any conflicts of
interest in this research. The mention of software packages is not an
endorsement of those packages. Susan VanderPlas and Tyler Wiederich
produced all material with the exception of all mentioned software
packages.

\newpage

\section{Abstract}\label{abstract}

A key component of statistics courses is to teach students how to
interpret data visualizations. Although many research-based
recommendations exist for creating graphs, the technological advances
for creating such graphs have outpaced studies that evaluate their
effectiveness, especially when considering 3D graphs. Here, we describe
a process for integrating an experiment on 3D graphs as part of a
project on statistical investigations for students enrolled in an
introductory statistics course and gathered responses as students
reflected on their experience of being a participant in the experiment
and then being a reviewer of empirical evidence about the experiment A
total of 82 students participated in our graphics project. As
participants in an experiment, they displayed a pattern of not fully
grasping research objectives as \DIFdelbegin \DIFdel{evidence }\DIFdelend \DIFaddbegin \DIFadd{evidenced }\DIFaddend by widely varied responses.
However, as reviewers of material from a pilot study of the same design,
they tended to gain a clearer understanding \DIFdelbegin \DIFdel{about }\DIFdelend \DIFaddbegin \DIFadd{of }\DIFaddend the purpose of the
experiment and its role in the realm of data visualizations by correctly
interpreting an extended abstract and video presentation of the pilot
study. The project we presented to students shows promise as an
educational tool for helping students gain a more holistic view of
statistical research, which is important in both the contributions to
data visualizations and the education of statistics.

\section{Introduction}\label{introduction}

The education of statistics emphasizes the use of real data and its
application in answering research questions. Students in introductory
statistics courses are mainly exposed to elementary methods and textbook
examples that demonstrate the application of these methods, which is
used, in part, because of the emphasis on teaching students to think
statistically using real data (\DIFdelbegin \DIFdel{\mbox{%DIFAUXCMD
\citeproc{ref-carver}{Carver, College,
and Everson 2016}}\hskip0pt%DIFAUXCMD
}\DIFdelend \DIFaddbegin \DIFadd{Carver, College, and Everson 2016}\DIFaddend ). Many
textbooks, such as Tintle et al. (\DIFdelbegin \DIFdel{\mbox{%DIFAUXCMD
\citeproc{ref-Tintle2021}{2021}}\hskip0pt%DIFAUXCMD
}\DIFdelend \DIFaddbegin \DIFadd{2021}\DIFaddend ), use scenarios and ask students
to perform the corresponding inferential test. This process is then
repeated over the course material without much deviation in the
instructional method. In some cases, however, students can participate
and benefit from well-designed classroom experiments (\DIFdelbegin \DIFdel{\mbox{%DIFAUXCMD
\citeproc{ref-mcgowan2011}{McGowan 2011}}\hskip0pt%DIFAUXCMD
}\DIFdelend \DIFaddbegin \DIFadd{McGowan 2011}\DIFaddend ).
This process can expose students to concepts such as randomization and
let students see the specifics of experimental design through their
participation. Loy (\DIFdelbegin \DIFdel{\mbox{%DIFAUXCMD
\citeproc{ref-loy2021a}{2021}}\hskip0pt%DIFAUXCMD
) have }\DIFdelend \DIFaddbegin \DIFadd{2021) has }\DIFaddend demonstrated that student participants
often recalled their experiment in later concepts, showing some evidence
that students can benefit from the hands-on experience.

In addition to using experiential learning broadly, a key aspect in the
statistics classroom is teaching students to interpret data through
graphs. Nearly 40 years ago, Cleveland and McGill (\DIFdelbegin \DIFdel{\mbox{%DIFAUXCMD
\citeproc{ref-cleveland1984}{1984}}\hskip0pt%DIFAUXCMD
}\DIFdelend \DIFaddbegin \DIFadd{1984}\DIFaddend ) began the
process of establishing good practices for making graphs. While their
findings have been replicated (\DIFdelbegin \DIFdel{\mbox{%DIFAUXCMD
\citeproc{ref-heer2010}{Heer and Bostock 2010}}\hskip0pt%DIFAUXCMD
}\DIFdelend \DIFaddbegin \DIFadd{Heer and Bostock 2010}\DIFaddend ), many areas in
data visualization remain underdeveloped that can benefit from the
framework used by Cleveland and McGill (\DIFdelbegin \DIFdel{\mbox{%DIFAUXCMD
\citeproc{ref-cleveland1984}{1984}}\hskip0pt%DIFAUXCMD
}\DIFdelend \DIFaddbegin \DIFadd{1984}\DIFaddend ). One such area is the
projections of 3D data visualizations. The current mantra is to avoid 3D
graphs when possible and studies around the 1990s \DIFdelbegin \DIFdel{(\mbox{%DIFAUXCMD
\citeproc{ref-barfield1989}{Barfield and Robless 1989}}\hskip0pt%DIFAUXCMD
;
\mbox{%DIFAUXCMD
\citeproc{ref-fisher1997}{Fisher, Dempsey, and Marousky 1997}}\hskip0pt%DIFAUXCMD
) }\DIFdelend seem to provide some
valid skepticism of their use \DIFaddbegin \DIFadd{(Barfield and Robless 1989; Fisher,
Dempsey, and Marousky 1997)}\DIFaddend . Barfield and Robless (\DIFdelbegin \DIFdel{\mbox{%DIFAUXCMD
\citeproc{ref-barfield1989}{1989}}\hskip0pt%DIFAUXCMD
}\DIFdelend \DIFaddbegin \DIFadd{1989}\DIFaddend ) showed that
participants were sometimes more accurate using 3D graphs than 2D
graphs, depending on the participant's experience level. However,
participants were generally more confident with their answers for 2D
graphs. Fisher, Dempsey, and Marousky (\DIFdelbegin \DIFdel{\mbox{%DIFAUXCMD
\citeproc{ref-fisher1997}{1997}}\hskip0pt%DIFAUXCMD
}\DIFdelend \DIFaddbegin \DIFadd{1997}\DIFaddend ) observed a similar
preference for 2D graphs over 3D graphs when extracting information, but
found no preference for visual appeal for the type of graph. While these
results provided valid skepticism toward the use of 3D graphs, the
results are generalized to the projections of the 3D graphs. The area of
``true'' 3D graphs is largely unexplored, but advances in technology
allow for easier access to explore the 3D projections of these graphs.
Kraus et al. (\DIFdelbegin \DIFdel{\mbox{%DIFAUXCMD
\citeproc{ref-kraus2020}{2020}}\hskip0pt%DIFAUXCMD
}\DIFdelend \DIFaddbegin \DIFadd{2020}\DIFaddend ) explored the 3D projections with the use of virtual
reality and found that participants were more accurate at extracting
values from 3D heatmaps at the expense of needing more time than 2D
alternatives.

Because the use of ``true'' 3D graphs remains largely unexplored, it
provides a unique opportunity to be used as an experiential learning
opportunity for statistics students. Not only can students benefit from
the exposure to different graph types, including ``true'' 3D graphs, but
they can also experience a more authentic method of teaching, which is
more likely to be beneficial to reinforce statistical ways of thinking.

In this paper, we used an experiential learning project that employed
different graph types, including ``true'' 3D graphs, in an introductory
statistics classroom environment and \DIFdelbegin \DIFdel{describe }\DIFdelend \DIFaddbegin \DIFadd{described }\DIFaddend its potential application
as an educational tool. We presented students with a series of modules
where they first participated in our experiment on different graph
\DIFdelbegin \DIFdel{types}\DIFdelend \DIFaddbegin \DIFadd{projections}\DIFaddend , followed by acquiring knowledge on the purpose of the
experiment by reading an extended abstract and watching a presentation
of a pilot study of the same experiment.

\section{Methods}\label{methods}

\DIFaddbegin \subsection{\DIFadd{Overview and Motivation}}\label{overview-and-motivation}

\DIFaddend We introduced students enrolled in Introduction to Statistics (STAT 218)
at the University of Nebraska-Lincoln to a graphics project that
contained an experiment and progressively \DIFdelbegin \DIFdel{revealing }\DIFdelend \DIFaddbegin \DIFadd{revealed }\DIFaddend components that
illustrate the experiment's research objectives. The project started by
providing students with minimal information about the research
objectives before revealing the scope of the experiment through an
extended abstract and presentation. The goal of the graphics project was
to observe how students think statistically about experiments as both
participants and research consumers. Students were provided with mostly
open-ended prompts throughout the graphics project, which allowed us to
freely explore common themes in student responses without the limitation
of preset choices.

\subsection{Experiential Learning}\label{experiential-learning}

The graphics project was split into two main components regarding the
interaction of students with the material. In this section, we will
discuss student interactions with the experiment. Students were first
presented with a series of four modules presented from the role of
research participation. These modules contained the informed consent
form, pre-experiment reflection, experiment participation, and
post-experiment reflection. Within the informed consent module, students
were informed that their data would be anonymized and that the
experiment was \DIFdelbegin \DIFdel{carried out in accordance with }\DIFdelend \DIFaddbegin \DIFadd{exempt from }\DIFaddend the institutional review board (Project ID:
22579). While all students were given the option to participate in the
graphics project, we were only able to collect responses when informed
consent was obtained and if the student was 19 years of age \DIFaddbegin \DIFadd{or older}\DIFaddend . In
the pre-experiment reflection, students were asked to write a paragraph
about how they think the process of scientific investigation looks from
the perspective of researchers and the general public. The experiment
participation module asked students to paste the code generated upon
completion of the graphics experiment, which is detailed in the next
section. The generated code serves as a basic check to indicate whether
or not students fully participated in the experiment. For the
post-experiment reflection, students were asked five questions about the
purpose of the experiment. These include questions on the hypotheses
being tested, sources of error, variables of interest, and elements of
experimental design.

After completing the experiment reflections, students moved to the
reflection of the overall research objectives. Students were first
directed to read an unpublished two-page extended abstract that we
submitted as a contributed paper for the Symposium on Data Science \&
Statistics (SDSS). The extended abstract outlined the experiment's
purpose and procedures, but not the results from our initial pilot
study. After reading the extended abstract, students were asked to write
a paragraph about what they found clearer about the experiment's purpose
than when they were a participant. Finally, students were directed to
watch a 12-minute pre-recorded presentation based on an abbreviated
version given at SDSS (\DIFdelbegin \DIFdel{\mbox{%DIFAUXCMD
\citeproc{ref-wiederich2023}{Wiederich 2023}}\hskip0pt%DIFAUXCMD
}\DIFdelend \DIFaddbegin \DIFadd{Wiederich 2023}\DIFaddend ). The video contained the same
material as the extended abstract and included the results from our
pilot study. The presentation reflection asked students four questions
about the experiment and how information was presented differently than
in the extended abstract.

Except for the informed consent and experiment participation modules,
all student responses were open-ended. Each question and its
corresponding module can be found in Table 1.

Instructors for STAT 218 were recruited for Summer 2023 and Fall 2023 to
administer the graphics project into their classroom. The instructors
were given the option of administering the project as coursework
material or extra credit, along with the liberty of grading at their
\DIFdelbegin \DIFdel{own
discretion. }\DIFdelend \DIFaddbegin \DIFadd{discretion. Instructors for the June-July Summer 2023 courses did not
have the abstract or presentation modules.
}\DIFaddend 

\DIFaddbegin \subsection{\DIFadd{Data Analysis}}\label{data-analysis}

\DIFadd{Since nearly all of the student responses to the project modules are
open-ended, the analysis of the project modules is qualitative. We will
selectively extract student responses that demonstrate variability and
repetitive themes in their understanding of the graphics experiment. For
all modules, except the Post-Experiment reflection, students are asked
to comment on the experiment without an objectively correct response.
}

\DIFadd{For paragraph responses in the Pre-Experiment and Abstract reflections,
bigram plots are used as a graphical analysis to display word pairs
after removing stop words (e.g., ``the'' and ``and''). These word pairs
help illustrate common themes that students wrote about in their longer
prompts. In the Post-Experiment reflection, the prompts have objectively
correct answers but would require a subjective aggregation to indicate
the level of correctness. Instead, we opt to select at least one student
response that we consider to be ``most correct'' and a couple of other
responses that are either partially correct or entirely incorrect to
illustrate the variability of the responses. The results of the graphics
experiment are presented by Wiederich and VanderPlas (n.d.) as an
extension of a larger study comparing the dimensionality and projections
of 2D and 3D bar charts.
}

\DIFaddend \subsection{Graphics Experiment}\label{graphics-experiment}

\subsubsection{Constructing Stimuli}\label{constructing-stimuli}

Based on Cleveland and McGill's seminal work (\DIFdelbegin \DIFdel{\mbox{%DIFAUXCMD
\citeproc{ref-cleveland1984}{1984}}\hskip0pt%DIFAUXCMD
}\DIFdelend \DIFaddbegin \DIFadd{1984}\DIFaddend ) on graphical
perception, participants were presented with a series of bar graphs
where two bars are marked with either a circle or a triangle. The
heights of the bars were chosen from the following equation:

\begin{equation}\phantomsection\label{eq-vals}{s_i=10\cdot 10^{(i-1)/12}, \qquad i=1,2,3,...,10}\end{equation}

where \(s_i\) represents a value given an integer \(i\) as defined
above. The values \(s_i\) from Equation~\ref{eq-vals} were then paired
such that the ratio of the smaller value to the larger value yield the
ratios of 0.178, 0.261, 0.383, 0.464, 0.562, 0.681, and 0.825. Each bar
graph has two groupings of five bars. Following the Type 1 and Type 3
graphs from the position-length experiment by Cleveland and McGill
(\DIFdelbegin \DIFdel{\mbox{%DIFAUXCMD
\citeproc{ref-cleveland1984}{1984}}\hskip0pt%DIFAUXCMD
}\DIFdelend \DIFaddbegin \DIFadd{1984}\DIFaddend ), the value pairs for each ratio were either placed in the first
grouping on the second and third bars (adjacent) \DIFdelbegin \DIFdel{, }\DIFdelend or placed on the second
bars in each grouping (separated).

To explore the effect of dimensionality and projection of the bar
graphs, we introduced the following plot types: 2D digital, 3D digital
(static), 3D digital (interactive), and 3D printed. There was no single
software package that could create all four plot types, so we carefully
constructed graphs from different software packages to be as similar as
possible. The 2D digital plots were rendered with the \texttt{ggplot2}
package (\DIFdelbegin \DIFdel{\mbox{%DIFAUXCMD
\citeproc{ref-ggplot2}{Wickham 2016}}\hskip0pt%DIFAUXCMD
}\DIFdelend \DIFaddbegin \DIFadd{Wickham 2016}\DIFaddend ). Microsoft Excel® was used to render the 3D
digital (static) plots. The 3D digital (interactive) and 3D printed
plots were created with OpenSCAD® (\DIFdelbegin \DIFdel{\mbox{%DIFAUXCMD
\citeproc{ref-kintelOpenSCADDocumentation2023}{Kintel 2023}}\hskip0pt%DIFAUXCMD
}\DIFdelend \DIFaddbegin \DIFadd{Kintel 2023}\DIFaddend ), where the generated
stereolithography (STL) files for the 3D digital (interactive) plots
were rendered with the \texttt{rgl} package (\DIFdelbegin \DIFdel{\mbox{%DIFAUXCMD
\citeproc{ref-rgl}{Murdoch and Adler 2023}}\hskip0pt%DIFAUXCMD
}\DIFdelend \DIFaddbegin \DIFadd{Murdoch and Adler 2023}\DIFaddend ).

\subsubsection{Experimental Design}\label{experimental-design}

With 56 treatment combinations, we opted to use an incomplete block
design to provide participants with 15-20 graphs. Kits of graphs were
constructed so that five of the seven ratios are equally represented,
resulting in 21 different kits. Within each kit, all graph types
appeared for each ratio and the comparison type was randomly assigned. A
visual layout of the experiment is shown in Figure 1. All kits received
a unique identifier and a set of instructions for accessing the
experiment.

A Shiny application (\DIFdelbegin \DIFdel{\mbox{%DIFAUXCMD
\citeproc{ref-shiny}{Chang et al. 2023}}\hskip0pt%DIFAUXCMD
}\DIFdelend \DIFaddbegin \DIFadd{Chang et al. 2023}\DIFaddend ) was designed to administer the
experiment. Students were directed to randomly select a kit of graphs
and visit the Shiny application's website linked on the instructions.
For students enrolled in the online sections of STAT 218, the website
was provided by the instructor and they were prompted in the application
to select that they were an online participant; selecting online
participation resulted in the 3D-printed plots being removed from the
set of graphs presented to the participant. After students provided a
kit identifier (if applicable), students were presented with graphs in a
randomized order. If the student marked that they were an online
participant, the 3D-printed graphs were removed from their experiment
lineup. Each graph asked the students to first identify the larger
marked bar and then to guess the height of the smaller marked bar if the
larger marked bar was 100 units tall using a slider widget. After
completing the experiment, a completion code was generated for students
to paste into the experiment participation module.

\DIFdelbegin \subsection{\DIFdel{Data Analysis}}%DIFAUXCMD
\addtocounter{subsection}{-1}%DIFAUXCMD
%DIFDELCMD < \label{data-analysis}
%DIFDELCMD < 

%DIFDELCMD < %%%
\DIFdel{Since nearly all of the student responses to the project modules are
open-ended, the analysis of the project modules is qualitative. We will
selectively extract student responses that demonstrate variability and
repetitive themes in their understanding of the graphics experiment. For
paragraph responses, bigram plots are used as a graphical analysis to
display word pairs after removing stop words (e.g., ``the'' and
``and''). These word pairs help illustrate common themes that students
wrote about in their longer prompts. The results of the graphics
experiment are presented by Wiederich and VanderPlas
(\mbox{%DIFAUXCMD
\citeproc{ref-wiederich}{n.d.}}\hskip0pt%DIFAUXCMD
) as an extension of a larger study
comparing the dimensionality and projections of 2D and 3D bar charts.
}%DIFDELCMD < 

%DIFDELCMD < %%%
\section{\DIFdel{Figure Legends}}%DIFAUXCMD
\addtocounter{section}{-1}%DIFAUXCMD
%DIFDELCMD < \label{figure-legends}
%DIFDELCMD < 

%DIFDELCMD < %%%
\DIFdel{Figure 1: Visual display of the experimental design for students who
participated in the 3D bar charts experiment. Kits of graphs were
created by first choosing five ratios from nine available options (1).
Each ratio then uses all graph types, with the exception of the
3D-printed graphs for online students (2). Finally, all graphs were
randomly assigned to have the marked bars as adjacent or separated (3).
}%DIFDELCMD < 

%DIFDELCMD < %%%
\DIFdelend \section{Results}\label{results}

\subsection{Recruitment of Students and
Instructors}\label{recruitment-of-students-and-instructors}

We recruited 3 instructors for \DIFaddbegin \DIFadd{the }\DIFaddend summer and fall semesters in 2023.
Each instructor offered the project as extra credit in their course and
student participation was entirely voluntary. A total of 82 students
participated in the project, and 9 students did not complete the \DIFdelbegin \DIFdel{project
completely }\DIFdelend \DIFaddbegin \DIFadd{entire
project }\DIFaddend (Table 2).

\subsection{Selected Responses from Experiment
Participation}\label{selected-responses-from-experiment-participation}

\subsubsection{Pre-Experiment
Reflection}\label{pre-experiment-reflection}

In the first stage of the project, \DIFaddbegin \DIFadd{before participating in the
experiment, }\DIFaddend students had limited information about the research
objectives and we recorded how students thought about scientific
research. \DIFdelbegin \DIFdel{Before to theexperiment, students }\DIFdelend \DIFaddbegin \DIFadd{Students were initially prompted with one question, ``What do
you think }{[}\DIFadd{the}{]} \DIFadd{process }{[}\DIFadd{of scientific investigation}{]} \DIFadd{looks
like, from the perspective of a researcher, compared to what it looks
like from the perspective of someone in the general public who is a
consumer of scientific results?'' The responses to this question allowed
us to gather a broad understanding of what students initially thought
about the research process. Students }\DIFaddend generally understood the purpose of
scientific research by connecting the ideas of hypothesis testing and
publishing results\DIFdelbegin \DIFdel{as demonstrated in the
Pre-Experiment bigram plot from the Pre-Experiment Reflection (Figure
2)}\DIFdelend . Students wrote about scientific research starting
from the place of a question, followed by conducting an experiment and
relaying the results to the public. For example, the most common phrase
groupings include variations of ``scientific research'', ``data
collection'', and ``peer review''. \DIFaddbegin \DIFadd{A bigram plot of student responses to
the Pre-Experiment reflection prompt is shown in Figure 2.
}\DIFaddend 

\subsubsection{Post-Experiment
Reflection}\label{post-experiment-reflection}

After participating in the experiment, students were provided prompts
asking about the goals of the research objectives\DIFaddbegin \DIFadd{, which were to
evaluate the errors of ratio judgments between 2D and 3D graph types
under different comparison conditions in a randomized block design}\DIFaddend . Some
students correctly identified parts of the questions asked in the
post-experiment reflection, but \DIFdelbegin \DIFdel{often missed the objective of comparing the accuracy of
ratio judgements of 2D and 3D graphs}\DIFdelend \DIFaddbegin \DIFadd{most students typically missed some or
most of what would be considered a correct answer}\DIFaddend .

\DIFdelbegin \DIFdel{When students were asked }\DIFdelend \DIFaddbegin \DIFadd{We first asked students }\DIFaddend ``What do you think the purpose of the
experiment was?'' \DIFdelbegin \DIFdel{, one student responded ``They could be trying to
determine how different genders, ages, etc. perceive the sizes of the
bars in the graph.Demographics could make a pretty significant
difference.}\DIFdelend \DIFaddbegin \DIFadd{For this question, a complete response would have
included the comparisons of ratio judgments of the projections for 2D
and 3D graphs. A student example of the correct response was ``I think
this experiment aimed to test if it was easier to compare two graphs in
2D or 3D.}\DIFaddend '' Another student \DIFdelbegin \DIFdel{responded }\DIFdelend \DIFaddbegin \DIFadd{missed the comparison of 2D to 3D graphs,
but was mostly correct with their response }\DIFaddend ``I think the purpose of this
experiment was for the researcher to gather data on how people perceive,
interpret, and understand 3D graphs.'' \DIFdelbegin \DIFdel{A third student correctly
commented ``I think this experiment aimed to test if it was easier to
compare two graphs in 2D or 3D.'' For this question, a complete response
would have included the comparisons of ratio judgements of the projections for 2D and 3D graphs. }\DIFdelend \DIFaddbegin \DIFadd{One student incorrectly thought
that we were testing differences in demographics by responding ``They
could be trying to determine how different genders, ages, etc. perceive
the sizes of the bars in the graph. Demographics could make a pretty
significant difference.''
}\DIFaddend 

We then asked students ``What hypotheses might the experimenters have
been testing?'' A correct response would include \DIFdelbegin \DIFdel{measuring differences
in accuracy of ratio judgements between 2D and 3D graphs. One student
correctly identified this by responding }\DIFdelend \DIFaddbegin \DIFadd{testing the differences
in errors of ratio judgments between graph types. One example of a
student providing a correct response was given by stating }\DIFaddend ``They might
have been testing if a 2D model is easier to estimate its relative size
to another when compared to a 3D model of it.'' Other students replied
with statements that would not be able to be measured from the
experiment, with one student responding ``How taking Statistics 218
effects how you can compare two groups'' and another student saying
``That 2d is preferred over 3d. It cleans up the data presentation.''

An important topic covered in STAT 218 is randomization, which we asked
students with \DIFaddbegin \DIFadd{the prompt }\DIFaddend ``What elements of experimental design, such as
randomization or the use of a control group, do you think were present
in the experiment? Why?'' \DIFaddbegin \DIFadd{In the experiment, randomization was performed
with the ordering and assignment of graphs into kits. }\DIFaddend One student
perfectly described randomization by responding with ``Randomization:
The survey used an experimental design where in the survey there were
different sets of 3D charts and maybe by a randomization process each
participant was shown a different set of charts to see the differences
in interpretations of the charts based on which set was assigned.
Control Group: Since this survey aimed to only understand how
participants interpret 3D charts without comparison to other chart
types, then no control group was needed.'' Another student was partially
correct with a response of ``Randomization was used because the ever
person got a different graph.'' Other students missed the \DIFdelbegin \DIFdel{utilization }\DIFdelend \DIFaddbegin \DIFadd{use }\DIFaddend of
randomization, with one student responding ``random students in the
stats class'' and another saying ``Randomization was not used because it
was offered as an extra credit assignment in class.''

\subsection{Selected Responses from Research
Reflections}\label{selected-responses-from-research-reflections}

In this set of reflections, students first read the extended abstract,
followed by watching the 12-minute presentation video. The abstract
unveiled the scope of the study to students, many of whom did not
realize the underlying complexities. Nearly all students responded with
statements about gaining clarity about the purpose \DIFaddbegin \DIFadd{of }\DIFaddend the experiment and
its role in testing the differences between 2D and 3D graphs. \DIFdelbegin \DIFdel{A bigram
plot of the student responses to the abstract reflection prompt is shown
in Figure 3. }\DIFdelend Students
commented on how they now understood the purpose of comparing 2D and 3D
graphs, and also the potential benefits that may stem from research,
such as graphical accessibility to the visually impaired. \DIFaddbegin \DIFadd{A bigram plot
of the student responses to the abstract reflection prompt is shown in
Figure 3.
}\DIFaddend 

Lastly, more than half of the students (78.5\%) responded that they
preferred the presentation over the extended abstract when asked ``If
you had to hear about this study using only the extended abstract or
only the presentation, which one would you prefer? Which one would be
better for determining whether the experiment was well designed?'' One
student responded ``I am a visual learner so I would have rather heard
about in through the presentation. It also broke down the steps which is
easier for me to understand. I think the presentation as a whole would
be better for determining how the experiment is designed.'' Another
student said ``Personally I like the abstract better. If I get confused
on something it is so much easier to go back and reread to understand
what is going on. If I ask myself questions about it, it is much easier
to go back and find answers to the questions as well.''

\section{Discussion}\label{discussion}

Taken together, our results support the idea that we achieved our goal
of providing students of an introductory statistics course with the
opportunity to reflect on active research. Students generally
appreciated the progressively revealing nature of the graphics project,
which is evident from the abstract and presentation reflections. When
provided with the post-experiment reflections, students often either
missed the research \DIFdelbegin \DIFdel{objective }\DIFdelend \DIFaddbegin \DIFadd{objectives }\DIFaddend of the experiment or had partially
correct responses. \DIFaddbegin \DIFadd{However, this was expected since the design of the
experiment exceeds the scope of experimental design taught in typical
introductory statistics courses.
}

\DIFaddend The abstract reflection received many responses indicating that students
had moments of realization about the true nature of our research goals,
which was further expanded \DIFdelbegin \DIFdel{in }\DIFdelend \DIFaddbegin \DIFadd{with }\DIFaddend the presentation reflection prompts.
Across all reflections, students were thoughtful, and sometimes amusing,
with their responses and \DIFdelbegin \DIFdel{that }\DIFdelend they were on the path of statistical thinking.

\DIFaddbegin \DIFadd{Having students participate in experiments is not uncommon, possibly
attributed to the readily available convenient samples within academia
and potential applications to introduce new course material (Margaret
1994). However, these studies typically end for students after
completing the experiment (McGowan 2011; Zacks et al. 1998; Fisher,
Dempsey, and Marousky 1997). In our study, we integrated the graphics
experiment as a course project with the addition of reviewing research
material.
}

\DIFaddend What we found was that the graphics project found success within the
recommendations of the GAISE College Report \DIFdelbegin \DIFdel{(\mbox{%DIFAUXCMD
\citeproc{ref-carver}{Carver, College, and Everson 2016}}\hskip0pt%DIFAUXCMD
}\DIFdelend \DIFaddbegin \DIFadd{by using real data to
illustrate the contextual purpose of scientific research (Carver,
College, and Everson 2016}\DIFaddend ). Students were able to demonstrate their
ability to think statistically through the series of reflections. The
graphics project made use of real data, along with data collection,
within the scope of an approachable and field-related topic, which
allowed \DIFdelbegin \DIFdel{for }\DIFdelend students to see how scientific research is conducted in the
field of statistics.

A limitation of our study is the use of open-ended responses that do not
objectively assess student learning. While the student responses were
useful in gathering insight, the responses are widely varied and do not
have direct comparisons of statistical thinking throughout the modules.
Another limiting factor is \DIFaddbegin \DIFadd{the }\DIFaddend tiered layering of convenience sampling,
with instructors being recruited before recruiting students, which
impacts the generalization of our findings. Nonetheless, the 82 students
who participated provided meaningful answers that displayed \DIFdelbegin \DIFdel{a }\DIFdelend \DIFaddbegin \DIFadd{some }\DIFaddend level
of statistical thinking throughout the graphics project.

Future studies could use a similar framework to conduct experiments on
more typical 3-dimensional data, such as heatmaps. The use of graphical
experiments in the classroom not only provides a readily available
convenience sample \DIFdelbegin \DIFdel{, }\DIFdelend but also adheres to the recommendations of the GAISE
College Report (\DIFdelbegin \DIFdel{\mbox{%DIFAUXCMD
\citeproc{ref-carver}{Carver, College, and Everson
2016}}\hskip0pt%DIFAUXCMD
}\DIFdelend \DIFaddbegin \DIFadd{Carver, College, and Everson 2016}\DIFaddend ). With the framework
we provided in this paper, we aim to make adjustments to further improve
the graphics experiment and corresponding project as an experiential
learning opportunity.

\section{Acknowledgments}\label{acknowledgments}

We would like to thank the Department of Statistics at \DIFaddbegin \DIFadd{the }\DIFaddend University of
Nebraska-Lincoln and the instructional team behind Introduction to
Statistics (STAT 218) for administering the experiment to students.

\section{Author Contributions}\label{author-contributions}

Susan VanderPlas \DIFdelbegin \DIFdel{created the framework of the graphics project,
submitted documentation to the Institutional Review Board for approval, and provided contributions for the code used to collect responses}\DIFdelend \DIFaddbegin \DIFadd{conceived the concept, designed the classroom project,
and assisted in data collection, data analysis, and manuscript
preparation}\DIFaddend . Tyler Wiederich designed the \DIFdelbegin \DIFdel{experiment, wrote the code for administering the
experiment, }\DIFdelend \DIFaddbegin \DIFadd{graphics experiment, }\DIFaddend recruited
and trained instructors, analyzed data, and wrote the manuscript.

\section{References}\label{references}

Link to journal citation style:
\href{https://www.tandfonline.com/action/authorSubmission?show=instructions&journalCode=ujse21\#refs}{here}

\phantomsection\label{refs}
\begin{CSLReferences}{1}{0}
\bibitem[\citeproctext]{ref-barfield1989}
Barfield, Woodrow, and Robert Robless. 1989. {``The Effects of Two- or
Three-Dimensional Graphics on the Problem-Solving Performance of
Experienced and Novice Decision Makers.''} \emph{Behaviour \&
Information Technology} 8 (5): 369--85.
\url{https://doi.org/10.1080/01449298908914567}.

\bibitem[\citeproctext]{ref-carver}
Carver, Robert, Stonehill College, and Michelle Everson. 2016.
{``Guidelines for Assessment and Instruction in Statistics Education
(GAISE) College Report.''}

\bibitem[\citeproctext]{ref-shiny}
Chang, Winston, Joe Cheng, JJ Allaire, Carson Sievert, Barret Schloerke,
Yihui Xie, Jeff Allen, Jonathan McPherson, Alan Dipert, and Barbara
Borges. 2023. {``Shiny: Web Application Framework for r.''}

\bibitem[\citeproctext]{ref-cleveland1984}
Cleveland, William S., and Robert McGill. 1984. {``Graphical Perception:
Theory, Experimentation, and Application to the Development of Graphical
Methods.''} \emph{Journal of the American Statistical Association} 79
(387): 531--54. \url{https://doi.org/10.1080/01621459.1984.10478080}.

\bibitem[\citeproctext]{ref-fisher1997}
Fisher, Samuel H., John V. Dempsey, and Robert T. Marousky. 1997.
{``Data Visualization: Preference and Use of Two-Dimensional and
Three-Dimensional Graphs.''} \emph{Social Science Computer Review} 15
(3): 256--63. \url{https://doi.org/10.1177/089443939701500303}.

\bibitem[\citeproctext]{ref-heer2010}
Heer, Jeffrey, and Michael Bostock. 2010. {``CHI '10: CHI Conference on
Human Factors in Computing Systems.''} In, 203--12. Atlanta Georgia USA:
ACM. \url{https://doi.org/10.1145/1753326.1753357}.

\bibitem[\citeproctext]{ref-kintelOpenSCADDocumentation2023}
Kintel, Marius. 2023. {``{OpenSCAD}. {OpenSCAD}.org.''} July 17, 2023.

\bibitem[\citeproctext]{ref-kraus2020}
Kraus, Matthias, Katrin Angerbauer, Juri Buchmüller, Daniel Schweitzer,
Daniel A. Keim, Michael Sedlmair, and Johannes Fuchs. 2020. {``CHI '20:
CHI Conference on Human Factors in Computing Systems.''} In, 1--14.
Honolulu HI USA: ACM. \url{https://doi.org/10.1145/3313831.3376675}.

\bibitem[\citeproctext]{ref-loy2021a}
Loy, Adam. 2021. {``Bringing Visual Inference to the Classroom.''}
\emph{Journal of Statistics and Data Science Education} 29 (2): 171--82.
\url{https://doi.org/10.1080/26939169.2021.1920866}.

\DIFaddbegin \bibitem[\citeproctext]{ref-margaret1994}
\DIFadd{Margaret, Mackisack. 1994. }{\DIFadd{``What Is the Use of Experiments Conducted
By Statistics Students?''}} \emph{\DIFadd{Journal of Statistics Education}} \DIFadd{2 (1):
2. }\url{https://doi.org/10.1080/10691898.1994.11910461}\DIFadd{.
}

\DIFaddend \bibitem[\citeproctext]{ref-mcgowan2011}
McGowan, Herle M. 2011. {``Planning a Comparative Experiment in
Educational Settings.''} \emph{Journal of Statistics Education} 19 (2):
4. \url{https://doi.org/10.1080/10691898.2011.11889612}.

\bibitem[\citeproctext]{ref-rgl}
Murdoch, Duncan, and Daniel Adler. 2023. {``Rgl: 3D Visualization Using
OpenGL.''}

\bibitem[\citeproctext]{ref-Tintle2021}
Tintle, Nathan, Beth L Chance, George W Cobb, Allan J Rossman, Soma Roy,
Todd Swanson, and Jill VanderStoep. 2021. \emph{Introduction to
Statistical Investigations}.

\bibitem[\citeproctext]{ref-ggplot2}
Wickham, Hadley. 2016. {``Ggplot2: Elegant Graphics for Data
Analysis.''}

\bibitem[\citeproctext]{ref-wiederich2023}
Wiederich, T. 2023. {``Evaluating Perceptual Judgements on 3D Printed
Bar Charts,''} May.
\url{https://twiedrw.github.io/3d_graphical_perception/SDSS_presentation/index.html}.

\bibitem[\citeproctext]{ref-wiederich}
Wiederich, T, and S VanderPlas. n.d. {``Placeholder: A Comparative
Experiment on Projections of 2D and 3D Bar Charts.''}

\DIFaddbegin \bibitem[\citeproctext]{ref-zacks1998}
\DIFadd{Zacks, Jeff, Ellen Levy, Barbara Tversky, and Diane J. Schiano. 1998.
}{\DIFadd{``Reading Bar Graphs: Effects of Extraneous Depth Cues and Graphical
Context.''}} \emph{\DIFadd{Journal of Experimental Psychology: Applied}} \DIFadd{4 (2):
119--38. }\url{https://doi.org/10.1037/1076-898X.4.2.119}\DIFadd{.
}

\DIFaddend \end{CSLReferences}

\newpage

\section{\DIFdelbegin \DIFdel{Figures}\DIFdelend \DIFaddbegin \DIFadd{Figure legends}\DIFaddend }\DIFdelbegin %DIFDELCMD < \label{figures}
%DIFDELCMD < %%%
\DIFdelend \DIFaddbegin \label{figure-legends}
\DIFaddend 

Figure 1: Visual display of the experimental design for students who
participated in the 3D bar charts experiment. Kits of graphs were
created by first choosing five ratios from nine available options (1).
Each ratio then uses all graph types, with the exception of the
\DIFdelbegin \DIFdel{3D
printed }\DIFdelend \DIFaddbegin \DIFadd{3D-printed }\DIFaddend graphs for online students (2). Finally, all graphs were
randomly assigned to have the marked bars as adjacent or separated (3).

Figure 2: Bigram of student responses \DIFaddbegin \DIFadd{(n=82) }\DIFaddend to the pre-experiment
prompt. Each line represents pairs of words that appeared together where
each pair occurred at least twice. Students generally understood that
science is about investigating research questions and collecting data.

Figure 3: Bigram of student responses \DIFaddbegin \DIFadd{(n=63) }\DIFaddend to the abstract reflection
prompt. Each line represents pairs of words that appeared together where
each pair occurred at least twice. Students generally \DIFdelbegin \DIFdel{understood that science
is about investigating research questions and collecting data}\DIFdelend \DIFaddbegin \DIFadd{focused on the
research objective of comparing 2D and 3D graphs}\DIFaddend .

\newpage

\section{Tables}\label{tables}

\begin{table}

\caption{\label{tab:unnamed-chunk-4}Questions provided to students in each project module.}
\centering
\fontsize{10}{12}\selectfont
\begin{tabu} to \linewidth {>{\raggedright\arraybackslash}p{9em}>{\raggedright\arraybackslash}p{6em}>{\raggedright\arraybackslash}p{25em}}
\toprule
Reflection & Question & Prompt\\
\midrule
Pre-Experiment & Q3 & In this class, you'll be learning about the process of scientific investigation. What do you think that process looks like, from the perspective of a researcher, compared to what it looks like from the perspective of someone in the general public who is a consumer of scientific results? Write a paragraph (at least 3-5 sentences) about how you think science happens.\\
\cmidrule{1-3}
 & Q5 & What do you think the purpose of the experiment was?\\
\cmidrule{2-3}
 & Q6 & What hypotheses might the experimenter have been testing?\\
\cmidrule{2-3}
 & Q7 & What sources of error are involved in this experiment?\\
\cmidrule{2-3}
 & Q8 & What variables were examined? For each variable, identify whether it was quantitative or categorical.\\
\cmidrule{2-3}
\multirow{-5}{*}{\raggedright\arraybackslash Post-Experiment} & Q9 & What elements of experimental design, such as randomization or the use of a control group, do you think were present in the experiment? Why?\\
\cmidrule{1-3}
Abstract & Q10 & What components of the experiment are clearer now than they were as a participant? What questions do you still have for the experimenter? Write 3-5 sentences reflecting on the abstract.\\
\cmidrule{1-3}
 & Q11 & How did the information you gained from the components of this project (participation, post-study reflection, extended abstract, presentation) differ?\\
\cmidrule{2-3}
 & Q12 & What components were emphasized in the presentation that weren't emphasized in the abstract? Why do you think that is?\\
\cmidrule{2-3}
 & Q13 & What critiques do you have of this study and its design? What would have made the study better?\\
\cmidrule{2-3}
\multirow{-4}{*}{\raggedright\arraybackslash Presentation} & Q14 & If you had to hear about this study using only the extended abstract or only the presentation, which one would you prefer? Which one would be better for determining whether the experiment was well designed?\\
\bottomrule
\end{tabu}
\end{table}
\DIFaddbegin 

\newpage
\DIFaddend 

\begin{table}[H]

\begin{threeparttable}
\caption{\label{tab:unnamed-chunk-5}Number of valid student participants by semester.}
\centering
\begin{tabular}[t]{lrr}
\toprule
Semester & Number of Sections & Number of Students\\
\midrule
Summer 2023 (May-June) & 1 & 17\\
Summer 2023 (July-Aug) & 1 & 23\\
Fall 2023 (May-June) & 1 & 42\\
\bottomrule
\end{tabular}
\begin{tablenotes}
\small
\item [] Students under 19 years of age or did not consent were exluded from data collection. To comply with IRB, no demographic information was collected to keep students anonymous with their reflections.
\end{tablenotes}
\end{threeparttable}
\end{table}



\end{document}
